
\documentclass[11pt,a4paper,sans]{moderncv} % Font sizes: 10, 11, or 12; paper sizes: a4paper, letterpaper, a5paper, legalpaper, executivepaper or landscape; font families: sans or roman
\usepackage{ragged2e}
\usepackage{footmisc}
\moderncvstyle{classic} % CV theme - options include: 'casual' (default), 'classic', 'oldstyle' and 'banking'
\moderncvcolor{green} % CV color - options include: 'blue' (default), 'orange', 'green', 'red', 'purple', 'grey' and 'black'

\usepackage[maxbibnames=99]{biblatex}
\defbibenvironment{bibliography}
{\list
	{\printtext[labelnumberwidth]{% label format from numeric.bbx
			\printfield{labelprefix}%
			\printfield{labelnumber}}}
	{\setlength{\topsep}{0pt}% layout parameters from moderncvstyleclassic.sty
		\setlength{\labelwidth}{\hintscolumnwidth}%
		\setlength{\labelsep}{\separatorcolumnwidth}%
		\leftmargin\labelwidth%
		\advance\leftmargin\labelsep}%
	\sloppy\clubpenalty4000\widowpenalty4000}
{\endlist}
{\item}
\addbibresource{references.bib}

\usepackage[scale=0.75]{geometry} % Reduce document margins
%\setlength{\hintscolumnwidth}{3cm} % Uncomment to change the width of the dates column
\setlength{\makecvtitlenamewidth}{10cm} % For the 'classic' style, uncomment to adjust the width of the space allocated to your name

%\newcommand{\showletter}{}
\newcommand{\showcv}{}



%----------------------------------------------------------------------------------------
%	NAME AND CONTACT INFORMATION SECTION
%----------------------------------------------------------------------------------------

\firstname{Daniel} % Your first name
\familyname{Schmitz} % Your last name

% All information in this block is optional, comment out any lines you don't need
\title{Curriculum Vitae}
\address{Dom-Pedro-Str. 9c}{D-80637 München}
\mobile{+49 1522 9230414}
%\phone{(000) 111 1112}
%\fax{(000) 111 1113}
\email{daniel.schmitz.public@gmail.com}
%\homepage{staff.org.edu/~jsmith}{staff.org.edu/$\sim$jsmith} % The first argument is the url for the clickable link, the second argument is the url displayed in the template - this allows special characters to be displayed such as the tilde in this example
%\extrainfo{additional information}
%\photo[70pt][0pt]{pictures/PP-Schmitz-741-druck-color.jpg} % TPP-PP-Schmitz-692-druck-color.jpghe first bracket is the picture height, the second is the thickness of the frame around the picture (0pt for no frame)
%\quote{"A witty and playful quotation" - John Smith}

%----------------------------------------------------------------------------------------

\begin{document}
	
	%----------------------------------------------------------------------------------------
	%	COVER LETTER
	%----------------------------------------------------------------------------------------
	
	% To remove the cover letter, comment out this entire block
	\ifdefined\showletter
	\clearpage
	\enlargethispage{12pt}
	\recipient{IMPRS for Brain and Behavior}{caesar\\
		Ludwig-Erhard-Allee 2\\53175 Bonn} % Letter recipient
	\date{\today} % Letter date
	\opening{Dear Sir or Madam,} % Opening greeting
	\closing{Yours sincerely,} % Closing phrase
	%\enclosure[Attached]{curriculum vit\ae{}} % List of enclosed documents
	
	\makelettertitle % Print letter title
	\justify
	I am writing to apply for a position in your doctoral program.
	I am currently completing my master's degree in bioinformatics at the Ludwig-Maximilians-Universität and Technical University of Munich and expect to graduate in early 2019.
	
	During my studies in the field of bioinformatics, I could familiarize myself with a wide array of computational methods enabling modern biological research.
	I am proficient in statistical analysis using R and Python with a focus on the evaluation of biochemical and genomic data such as next generation sequencing and differential expression analysis as well as the biological reasoning behind them.
	Additionally, I could gain insights on the modeling and prediction of molecules, components and processes not only on a cellular level but also in the context of systems biology.
	Thanks to my exceptional performance during my studies, I was awarded a membership in best.in.tum, which is a promotion program of the Technical University's Department of Informatics for the best 2\% of students.
	
	During my time in Munich, I started working at the Center for Human Genetics and Laboratory Diagnostics (MVZ), where I continue to be employed to this day.
	My work there allowed me to gain experience with the application of NGS in a diagnostic setting, especially in the area of rare diseases.
	In this interdisciplinary setting where I work at the intersection of multiple disciplines, including genomics, physiology and computer science, I was introduced to best practices for the evaluation of high-throughput data and applied my acquired knowledge directly to real clinical information.
	In addition to working with already established techniques, I was tasked with improving on the analysis pipeline.
	This included the development of applications enabling and simplifying the integration of multiple data preparation and analysis tools as well as the implementation of efficient, domain-specific evaluation methods.
	
	My current research is conducted as part of the Multiple Integration of Data Annotation Study (MIDAS), which is performed by the MVZ in tandem with the TUM School of Medicine.
	MIDAS explores the possibility of optimizing the diagnostic evaluation of high-throughput data when dealing with rare diseases by integrating additional data sets with the results of already established NGS analyses.
	This project provided the context of my master's thesis \textit{Syndrome Prediction by Phenotype Similarity and Its Application to Neurodevelopmental Disorders}, which involves the development of a system aiding in the diagnosis of rare diseases by identifying patients similar to each other.
	The thesis is focused on patients with developmental disorders and intellectual disabilities who could not receive a diagnosis through conventional methods and therefore had to undergo whole exome analysis.
	The system identifies similar patients by comparing their symptoms---provided as terms from the Human Phenotype Ontology---and suggests affected genes based on the results of these queries.
	
	My work with developmental and intellectual disorders piqued my interest in neurological science.
	After having worked in diagnostics and applying the knowledge gathered by neuroscientists all around the world, I am excited to contribute to the research making the identification and treatment of these disorders possible.
	My research interests lie in the biological backgrounds of cognition and memory and how their development and function are affected by neurological disorders. 
	I am sure, my prior studies and experience make me a great fit for your program.
	
	Thank you for considering my application.
	I am looking forward to hearing from you.
	
	\makeletterclosing % Print letter signature
	
	
	\fi
	%----------------------------------------------------------------------------------------
	%	CURRICULUM VITAE
	%----------------------------------------------------------------------------------------
	\ifdefined\showcv
	\newpage
	\makecvtitle % Print the CV title
	
	%----------------------------------------------------------------------------------------
	%	EDUCATION SECTION
	%----------------------------------------------------------------------------------------
	\vspace{-20pt}
	%\cvitem{Date of Birth}{May 3\textsuperscript{rd}, 1995}
	%\cvitem{Place of Birth}{Berlin}
	
	\section{Education}
	
	\cventry{since 2017}{M.Sc. Bioinformatics}{Ludwig-Maximilians-Universität and Technical University of Munich}{}{Graduation by April 2019. Current Grade: 1.6}{}  % Arguments not required can be left empty
	\cventry{2013--2017}{B.Sc. Bioinformatics}{Ludwig-Maximilians-Universität and Technical University of Munich}{}{Grade: 1.6}{}
	\cventry{2005--2013}{Abitur}{Friedrich-Schiller-Gymnasium Königs Wusterhausen}{}{Grade: 1.0}{}
	
	\subsection{Master's Thesis}
	
	\cvitem{Title}{\textit{Syndrome Prediction by Phenotype Similarity and Its Application to Neurodevelopmental Disorders}}
	\cvitem{Supervisors}{Professor Dimitrij Frishman, Chair of Bioinformatics, TUM}
	%\vspace{-6pt}
	\cvitem{}{Dr. Sebastian H. Eck, Center for Human Genetics and Laboratory Diagnostics}
	\cvitem{Description}{I developed a system assisting geneticists in identifying developmental disorders using phenotypic annotations.
		Annotations are based on terms from the Human Phenotype Ontology. Suggestions are made by comparing each new case to patients with an already existing diagnosis.}
	
	%----------------------------------------------------------------------------------------
	%	WORK EXPERIENCE SECTION
	%----------------------------------------------------------------------------------------
	
	\section{Practical Experience}
	
	\subsection{Vocational}
	
	\cventry{since 2014}{Software Developer/Student Employee}{Center for Human Genetics and Laboratory Diagnostics}{Planegg}{}{%
		Assisted in performing genetic analyses in a diagnostic setting focusing on rare diseases and neurodevelopmental disorders and developed software improving their outcomes and lowering their costs.
		\\
		\\
		Detailed Achievements:
		\begin{itemize}
			\item Supported general diagnostic NGS analyses
			\item Improved downstream analysis by implementing efficient statistical methods
			\item Developed applications streamlining the analytic workflow by simplifying the integration of multiple diagnostic tools and databases
			\item Designed and implemented software for the Multiple Integration of Data Annotation Study, exploring the evaluation of NGS data in a diagnostic setting by integrating data from multiple sources.
		\end{itemize}
	}
	\enlargethispage{20pt}
	%------------------------------------------------
	
	%------------------------------------------------
	
	\subsection{Voluntary}
	
	\cventry{since 2017}{Head of Membership Administration}{Verein für Corpsstudentische Geschichtsforschung e.V}{}{}{
		Management of the membership database of a German society concerned with academic history counting 1,100 members.
	}
	
	%----------------------------------------------------------------------------------------
	%	AWARDS SECTION
	%----------------------------------------------------------------------------------------
	
	\section{Awards}
	
	\cventry{2018}{Member of best.in.tum}{}{}{}{Promotion programme for the top 2\% of students at the TUM Department of Informatics}
	%----------------------------------------------------------------------------------------
	%	COMPUTER SKILLS SECTION
	%----------------------------------------------------------------------------------------
	
	\nocite{gpms, midas}
	\printbibliography[title={Publications}]
	
	%\section{Publications}
	%\cventry{2018}{MIDAS GPMS: A flexible gene panel management system}{Poster at ESHG}{Milan}{}{}
	%\cventry{2015}{A software tool for data Integration in a diagnostic laboratory}{Poster at Genome Informatics}{Cold Spring Harbor, NY}{}{}
	
	\section{Programming \& Data Analysis Skills}
	
	\cvitem{Basic}{Rust, C}
	\cvitem{Intermediate}{Perl, UNIX shell}
	\cvitem{Advanced}{Java, R, Python, SQL, Microsoft Excel, \LaTeX}
	
	%----------------------------------------------------------------------------------------
	%	LANGUAGES SECTION
	%----------------------------------------------------------------------------------------
	
	\section{Languages}
	
	
	\cvitemwithcomment{German}{Mothertongue}{}
	\cvitemwithcomment{English}{Fluent}{}
	\cvitemwithcomment{Spanish}{Intermediate}{}
	\cvitemwithcomment{French}{Basic}{}
	
	%----------------------------------------------------------------------------------------
	%	INTERESTS SECTION
	%---------------------------------------------------------------------------------------
	%----------------------------------------------------------------------------------------
	\section{Referees}
	\cvdoubleitem{}{\textbf{Prof. Dr. Volker Heun}\\
		LFE Bioinformatik\\
		Institut für Informatik\\
		Ludwig-Maximilians-Universität\\
		Amalienstr. 17\\
		80333 München\\
		\textsc{Germany}\\[5pt]
		\href{mailto:Volker.Heun@bio.ifi.lmu.de}{Volker.Heun@bio.ifi.lmu.de}}{}{\textbf{Dr. Sebastian Eck}\\
		Center for Human Genetics and Laboratory Diagnostics\\
		Lochhamer Str. 29\\
		82152 Martinsried\\
		\textsc{Germany}\\[5pt]
		\href{mailto:Sebastian.Eck@medizinische-genetik.de}{Sebastian.Eck@medizinische-genetik.de}}
	\fi
\end{document}