%%%%%%%%%%%%%%%%%
% This is an sample CV template created using altacv.cls
% (v1.7, 9 August 2023) written by LianTze Lim (liantze@gmail.com). Compiles with pdfLaTeX, XeLaTeX and LuaLaTeX.
%
%% It may be distributed and/or modified under the
%% conditions of the LaTeX Project Public License, either version 1.3
%% of this license or (at your option) any later version.
%% The latest version of this license is in
%%    http://www.latex-project.org/lppl.txt
%% and version 1.3 or later is part of all distributions of LaTeX
%% version 2003/12/01 or later.
%%%%%%%%%%%%%%%%

%% Use the "normalphoto" option if you want a normal photo instead of cropped to a circle
% \documentclass[10pt,a4paper,normalphoto]{altacv}

\documentclass[9pt,a4paper,ragged2e,withhyper]{altacv}
%% AltaCV uses the fontawesome5 and packages.
%% See http://texdoc.net/pkg/fontawesome5 for full list of symbols.

% Change the page layout if you need to
\geometry{left=1.25cm,right=1.25cm,top=1.5cm,bottom=1.5cm,columnsep=0.5cm}

% The paracol package lets you typeset columns of text in parallel
\usepackage{paracol}

% Change the font if you want to, depending on whether
% you're using pdflatex or xelatex/lualatex
% WHEN COMPILING WITH XELATEX PLEASE USE
% xelatex -shell-escape -output-driver="xdvipdfmx -z 0" sample.tex
\ifxetexorluatex
  % If using xelatex or lualatex:
  \setmainfont{Roboto Slab}
  \setsansfont{Lato}
  \renewcommand{\familydefault}{\sfdefault}
\else
  % If using pdflatex:
  \usepackage[rm]{roboto}
  \usepackage[defaultsans]{lato}
  % \usepackage{sourcesanspro}
  \renewcommand{\familydefault}{\sfdefault}
\fi

% Change the colours if you want to
\definecolor{SlateGrey}{HTML}{2E2E2E}
\definecolor{LightGrey}{HTML}{666666}
\definecolor{DarkPastelRed}{HTML}{450808}
\definecolor{PastelRed}{HTML}{8F0D0D}
\definecolor{GoldenEarth}{HTML}{E7D192}
\colorlet{name}{black}
\colorlet{tagline}{PastelRed}
\colorlet{heading}{DarkPastelRed}
\colorlet{headingrule}{GoldenEarth}
\colorlet{subheading}{PastelRed}
\colorlet{accent}{PastelRed}
\colorlet{emphasis}{SlateGrey}
\colorlet{body}{LightGrey}

% Change some fonts, if necessary
\renewcommand{\namefont}{\Huge\rmfamily\bfseries}
\renewcommand{\personalinfofont}{\footnotesize}
\renewcommand{\cvsectionfont}{\LARGE\rmfamily\bfseries}
\renewcommand{\cvsubsectionfont}{\large\bfseries}


% Change the bullets for itemize and rating marker
% for \cvskill if you want to
\renewcommand{\cvItemMarker}{{\small\textbullet}}
\renewcommand{\cvRatingMarker}{\faCircle}
% ...and the markers for the date/location for \cvevent
% \renewcommand{\cvDateMarker}{\faCalendar*[regular]}
% \renewcommand{\cvLocationMarker}{\faMapMarker*}


% If your CV/résumé is in a language other than English,
% then you probably want to change these so that when you
% copy-paste from the PDF or run pdftotext, the location
% and date marker icons for \cvevent will paste as correct
% translations. For example Spanish:
% \renewcommand{\locationname}{Ubicación}
% \renewcommand{\datename}{Fecha}


%% Use (and optionally edit if necessary) this .tex if you
%% want to use an author-year reference style like APA(6)
%% for your publication list
% % When using APA6 if you need more author names to be listed
% because you're e.g. the 12th author, add apamaxprtauth=12
\usepackage[backend=biber,style=apa6,sorting=ydnt]{biblatex}
\defbibheading{pubtype}{\cvsubsection{#1}}
\renewcommand{\bibsetup}{\vspace*{-\baselineskip}}
\AtEveryBibitem{%
  \makebox[\bibhang][l]{\itemmarker}%
  \iffieldundef{doi}{}{\clearfield{url}}%
}
\setlength{\bibitemsep}{0.25\baselineskip}
\setlength{\bibhang}{1.25em}


%% Use (and optionally edit if necessary) this .tex if you
%% want an originally numerical reference style like IEEE
%% for your publication list
\usepackage[backend=biber,style=ieee,sorting=ydnt,defernumbers=true,maxbibnames=999]{biblatex}
%% For removing numbering entirely when using a numeric style
\setlength{\bibhang}{1.25em}
\DeclareFieldFormat{labelnumberwidth}{\makebox[\bibhang][l]{\itemmarker}}
\setlength{\biblabelsep}{0pt}
\defbibheading{pubtype}{\cvsubsection{#1}}
\renewcommand{\bibsetup}{\vspace*{-\baselineskip}}
\AtEveryBibitem{%
  \iffieldundef{doi}{}{\clearfield{url}}%
}


%% sample.bib contains your publications
\addbibresource{sample.bib}

\begin{document}
\name{Daniel Schmitz}
\tagline{Bioinformatician | LGBTQ+ Activist}
%% You can add multiple photos on the left or right
%\photoR{2.8cm}{Globe_High}
% \photoL{2.5cm}{Yacht_High,Suitcase_High}

\personalinfo{%
  % Not all of these are required!
  \email{daniel@schmytzi.com}
  \phone{+46 70-047 06 04}
  \mailaddress{Väktargatan 56E}
  \location{754 22 Uppsala, Sweden}
  \homepage{www.schmytzi.com}
%  \twitter{@twitterhandle}
  \linkedin{daniel-schmitz-ab0089219}
  \github{schmytzi}
  \orcid{0000-0003-4480-891X}
  %% You can add your own arbitrary detail with
  %% \printinfo{symbol}{detail}[]
  % \printinfo{\faPaw}{Hey ho!}[https://example.com/]

  %% Or you can declare your own field with
  %% \NewInfoFiled{fieldname}{symbol}[optional hyperlink prefix] and use it:
  % \NewInfoField{gitlab}{\faGitlab}[https://gitlab.com/]
  % \gitlab{your_id}
  %%
  %% For services and platforms like Mastodon where there isn't a
  %% straightforward relation between the user ID/nickname and the hyperlink,
  %% you can use \printinfo directly e.g.
  % \printinfo{\faMastodon}{@username@instace}[https://instance.url/@username]
  %% But if you absolutely want to create new dedicated info fields for
  %% such platforms, then use \NewInfoField* with a star:
  % \NewInfoField*{mastodon}{\faMastodon}
  %% then you can use \mastodon, with TWO arguments where the 2nd argument is
  %% the full hyperlink.
  % \mastodon{@username@instance}{https://instance.url/@username}
}

\makecvheader
%% Depending on your tastes, you may want to make fonts of itemize environments slightly smaller
\AtBeginEnvironment{itemize}{\small}

\cvsection{Professional Summary}
I recently earned a PhD in medical science with a focus on bioinformatics and genomics in large cohort studies, and I have over four years of experience in medical genetics.
My expertise includes software development, high-performance computing, and statistics, as well as proficiency in next-generation sequencing (NGS) technologies for research and clinical use.
I have successfully completed projects through collaboration with diverse teams across various professional and personal backgrounds

\columnratio{0.6}
% Start a 2-column paracol. Both the left and right columns will automatically
% break across pages if things get too long.
\begin{paracol}{2}
\cvsection{Professional Experience}
\cvevent{Researcher}{Uppsala University\\ Department of Immunology, Genetics and Pathology}{Mar 2024 -- Present}{Uppsala, Sweden}
\begin{itemize}
  \item Prepared material for new precision medicine course
  \item Contributed to supervision of Master project students
  \item Advised on machine-learning approaches for research into effects of medication on plasma protein measurements
\end{itemize}
\divider

\cvevent{Visiting PhD Student}{National Cancer Institute\\ Laboratory for Bioinformatics and Computational Biology}{Sep 2023 -- Nov 2023}{Rio de Janeiro, Brazil}
\begin{itemize}
  \item International research collaboration on ovarian cancer using spatial transcriptomics
\end{itemize}

\divider

\cvevent{PhD Candidate}{Uppsala University\\ Department of Immunology, Genetics and Pathology}{Sep 2019 -- Feb 2024}{Uppsala, Sweden}
\begin{itemize}
  \item Realized and designed research projects in genomics involving large cohorts and novel technologies in a highly collaborative and diverse group
  \item Gave 4 poster presentations and 1 flash talk at international conferences
  \item 340 hours of teaching and supervision of Bachelor and Master students
\end{itemize}
\divider

\cvevent{Bioinformatics Assistant, Software Developer}{Center for Human Genetics and Laboratory Diagnostics}{Oct 2014 -- Jul 2019}{Planegg, Germany}
\begin{itemize}
  \item Developed patient information management system integrating NGS workflows
  \item Collaborated closely with other departments involved in diagnostics
  \item Optimized workflows for improved cost and outcomes
\end{itemize}

\cvsection{Voluntary Experience}

\cvevent{Chairperson}{RFSL Uppsala}{Mar 2022 -- Mar 2023}{Uppsala, Sweden}
\begin{itemize}
  \item Led the local chapter of Sweden's largest LGBTQ+ organization
  \item Represented organization to several news outlets
\end{itemize}

\divider

\cvevent{Secretary}{Medical PhD Student Board}{Jul 2021 -- Jun 2023}{Uppsala, Sweden}
\begin{itemize}
  \item Familiarized myself with bylaws and initiated their revision
  \item Prepared and documented meetings and ensured compliance to procedures
\end{itemize}

\switchcolumn
\cvsection{Most Proud of}
\cvachievement{\faDatabase}{SweGen T2T-CHM13 Release}{Reanalysed WGS data of 1,000 people using the novel reference T2T-CHM13}

\divider

\cvachievement{\faUsers}{Uppsala Pride 2022}{Oversaw the first pride festival after the COVID-19 pandemic (>3,000 attendees)}

\divider

\cvachievement{\faComments}{Queer Regina}{Initiated collaboration bringing local LGBTQ+ art to the theater stage}

\cvsection{Skills}

\cvtag{Project management}
\cvtag{NGS analysis}\\
\cvtag{Statistics}
\cvtag{Genomics}
\cvtag{Communication}\\
\smallskip\divider\smallskip

\cvtag{Software Development (R, Python, Java)}
\cvtag{Databases}
\cvtag{Documentation}
\cvtag{Nextflow}
\cvtag{Git}
\cvtag{\LaTeX}\\
\cvtag{High-performance computing}
\cvtag{Containerization}

\cvsection{Languages}

\cvskill{German}{5}
\smallskip
\cvskill{English}{4.5}
\smallskip
\cvskill{Swedish}{4}
\smallskip
\cvskill{Portuguese (BR)}{3} %% Supports X.5 values.

\cvsection{Education}

\cvevent{Ph.D.\ in Medical Science}{Uppsala University}{Sep 2019 -- Feb 2024}{}
Thesis: \href{https://urn.kb.se/resolve?urn=urn:nbn:se:uu:diva-517196}{\emph{Beyond GWAS: Novel Methods and Resources for Genetic Epidemiology}}

\divider

\cvevent{M.Sc.\ in Bioinformatics}{Ludwig-Maximilians-Universität,\\Technical University of Munich}{Oct 2017 -- Dec 2018}{}

\divider

\cvevent{B.Sc.\ in Bioinformatics}{Ludwig-Maximilians-Universität,\\Technical University of Munich}{Oct 2013 -- Aug 2017}{}

\switchcolumn

\newpage

\cvsection{Publications}

%% Specify your last name() and first name(s) as given in the .bib to automatically bold your own name in the publications list.
%% One caveat: You need to write \bibnamedelima where there's a space in your name for this to work properly; or write \bibnamedelimi if you use initials in the .bib
%% You can specify multiple names, especially if you have changed your name or if you need to highlight multiple authors.
\mynames{Schmitz/Daniel}
%% MAKE SURE THERE IS NO SPACE AFTER THE FINAL NAME IN YOUR \mynames LIST

\nocite{*}

\printbibliography[heading=pubtype,title={\printinfo{\faFile*[regular]}{Journal Articles}},type=article]

\switchcolumn

\cvsection{Grants}
\cvachievement{2023}{}{}
\cvachievement{}{Rönnows Resestipendium}{SEK 40,000}
\cvachievement{}{Anna Maria Lundin Resestipendium}{SEK 45,000}

\divider

\cvachievement{2022}{}{}
\cvachievement{}{Rudbergs, A och M}{SEK 79,441}
\cvachievement{}{Bergmark Travel}{SEK 34,605}

\divider

\cvachievement{2021}{}{}
\cvachievement{}{Anna Maria Lundin Resestipendium}{SEK 38,680}

\divider

\cvachievement{2020}{}{}
\cvachievement{}{Sederholms, utrikes}{SEK 19,000}

\end{paracol}
\end{document}
